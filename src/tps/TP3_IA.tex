\documentclass[a4paper,12pt]{article}
\usepackage[utf8]{inputenc}
\usepackage[T1]{fontenc}
\usepackage[margin=2.5cm]{geometry}
\usepackage[french]{babel}
\usepackage{hyperref}
\usepackage{listings}
\usepackage{xcolor}
\usepackage{fancyhdr}
\usepackage{textcomp} 

\lstset{
    basicstyle=\ttfamily\small,
    keywordstyle=\color{blue},
    stringstyle=\color{red!70!black},
    commentstyle=\color{green!50!black},
    breaklines=true,
    showstringspaces=false
}

\pagestyle{fancy}
\fancyhf{}
\fancyfoot[C]{M2 MIASHS}
\fancyfoot[R]{\thepage}

\title{Travaux Pratiques - Systemes de Recommandation Hybride, Classification (CNN) et Expressions Regulieres}
\author{Alexandre JOUSSET}
\date{\today}

\begin{document}

\maketitle

\section*{Introduction}
Ce TP propose trois parties pour découvrir de manière pratique :  
\begin{itemize}
\item Les systèmes de recommandation simples,
\item La classification d'images avec un CNN basique,
\item L'utilisation d'expressions régulières pour extraire des informations.
\end{itemize}

\section*{Objectifs généraux}
\begin{itemize}
\item Découvrir les méthodes de recommandation collaborative et basée sur le contenu.
\item Comprendre la construction et l'entraînement d'un CNN simple.
\item Pratiquer des regex pour extraire emails, téléphones, dates et URLs.
\end{itemize}

\section*{Matériel requis}
\begin{itemize}
\item Python 3.8+, Jupyter/Colab.
\item Bibliothèques : pandas, numpy, scikit-learn, surprise, tensorflow/keras, re.
\item Dataset : MovieLens 100k et Fashion-MNIST (inclus dans Keras).
\end{itemize}

\section*{Partie A : Recommandation simplifiée}
\subsection*{But}
Créer un système de recommandation simple combinant filtrage collaboratif et contenu.

\subsection*{Exercices}
\begin{enumerate}
\item Charger MovieLens 100k et créer \texttt{df\_ratings} et \texttt{movies\_df}.
\item Implémenter un filtrage collaboratif basique avec Surprise (user-based ou item-based).
\item Construire une recommandation simple basée sur les genres préférés d'un utilisateur.
\item Combiner les deux scores pour afficher les 5 meilleurs films recommandés.
\end{enumerate}

\subsection*{Questions de réflexion}
\begin{itemize}
\item Quels avantages/inconvénients des approches collaboratives et contenu ?
\item Pourquoi combiner les deux approches ?
\end{itemize}

\section*{Partie B : Classification d'images (CNN simplifié)}
\subsection*{But}
Classer les images du jeu Fashion-MNIST avec un CNN simple.

\subsection*{Exercices}
\begin{enumerate}
\item Charger et normaliser les images.
\item Créer un CNN simple (Conv -> Pool -> Dense).
\item Compiler et entraîner le modèle sur un petit nombre d'époques.
\item Évaluer accuracy sur le test et afficher quelques prédictions correctes/incorrectes.
\end{enumerate}

\subsection*{Questions de réflexion}
\begin{itemize}
\item Pourquoi normaliser les images ?
\item Comment savoir si le modèle apprend correctement ?
\end{itemize}

\section*{Partie C : Expressions régulières simples}
\subsection*{But}
Extraire emails, téléphones et URLs depuis un fichier texte.

\subsection*{Exercices}
\begin{enumerate}
\item Créer un fichier \texttt{sample\_text.txt} avec quelques emails, téléphones et URLs.
\item Écrire des regex simples pour extraire ces informations.
\item Tester vos regex avec des exemples pour vérifier qu'elles fonctionnent.
\end{enumerate}

\subsection*{Questions de réflexion}
\begin{itemize}
\item Quelles limites ont les regex pour valider des emails et URLs ?
\end{itemize}

\section*{Livrables}
\begin{itemize}
\item Notebooks pour chaque partie.
\item README simple expliquant comment exécuter le code.
\item Pour la partie recommandation : au moins 2 exemples d'utilisateurs.
\item Pour le CNN : afficher courbes et quelques prédictions.
\item Pour les regex : fichier texte et tests.
\end{itemize}


\section*{Date de rendu}
Le TP doit etre rendu avant le \textbf{8 octobre 2025} a 20h00.

\section*{Ressources et aides}
\begin{itemize}
\item MovieLens : \url{https://grouplens.org/datasets/movielens/} (utiliser la version 100k pour le TP si prefere)
\item Keras / TensorFlow documentation pour Fashion-MNIST.
\item Tutoriels sur Surprise (collaborative filtering) et sur TF-IDF / TfidfVectorizer.
\end{itemize}

\end{document}
