\documentclass[a4paper,12pt]{article}
\usepackage[utf8]{inputenc}
\usepackage[T1]{fontenc}
\usepackage[margin=2.5cm]{geometry}
\usepackage[french]{babel}
\usepackage{hyperref}
\usepackage{listings}
\usepackage{xcolor}
\usepackage{fancyhdr}
\usepackage{textcomp}

\lstset{
    basicstyle=\ttfamily\small,
    keywordstyle=\color{blue},
    stringstyle=\color{red!70!black},
    commentstyle=\color{green!50!black},
    breaklines=true,
    showstringspaces=false
}

\pagestyle{fancy}
\fancyhf{}
\fancyfoot[C]{M2 MIASHS}
\fancyfoot[R]{\thepage}

\title{Travaux Pratiques 4 - Automatisation et Tests avec Playwright}
\author{Alexandre JOUSSET}
\date{20 octobre 2025}

\begin{document}

\maketitle

\section*{Introduction}
Ce TP vous propose une approche pratique de l’automatisation de tests web avec \textbf{Playwright}, un outil moderne développé par Microsoft. Vous apprendrez à configurer un projet, interagir avec des éléments web, utiliser le \textit{Page Object Model}, simuler des scénarios complexes.

\section*{Objectifs}
\begin{itemize}
\item Maîtriser les bases de Playwright (installation, configuration, exécution)
\item Manipuler des locators et interactions utilisateur
\item Écrire des tests fiables avec assertions et debugging
\item Structurer son projet selon le modèle Page Object Model
\item Découvrir le mocking réseau et l’intégration CI/CD
\end{itemize}

\section*{Matériel requis}
\begin{itemize}
\item Node.js (version 18 ou supérieure)
\item Un éditeur de code (Visual Studio Code recommandé)
\item Git pour versionner le projet
\end{itemize}

\newpage
\section*{Partie A : Installation et premiers pas}
\subsection*{But}
Initialiser un projet Playwright et écrire un premier test simple.

\subsection*{Exercices}
\begin{enumerate}
\item Créez un dossier pour le projet et initialisez Playwright :
\begin{lstlisting}
mkdir tp4-playwright && cd tp4-playwright
npm init -y
npm init playwright@latest
\end{lstlisting}

\item Explorez la structure générée :
\begin{itemize}
\item \texttt{playwright.config.ts} : configuration globale
\item \texttt{tests/} : répertoire contenant les fichiers de tests
\item \texttt{package.json} : dépendances du projet
\end{itemize}

\item Créez un premier test simple :
\begin{lstlisting}
// tests/homepage.spec.ts
import { test, expect } from '@playwright/test';

test('page a le bon titre', async ({ page }) => {

});
\end{lstlisting}

\item Exécutez le test sur différents navigateurs :
\begin{lstlisting}
npx playwright test --project=chromium
npx playwright test --project=firefox
npx playwright test --project=webkit
\end{lstlisting}
\end{enumerate}

\subsection*{Questions de réflexion}
\begin{itemize}
\item Pourquoi tester sur plusieurs navigateurs est-il important ?
\item Comment Playwright gère-t-il automatiquement les temps d’attente entre actions ?
\end{itemize}

\newpage
\section*{Partie B : Interactions complexes et assertions avancées}
\subsection*{But}
Explorer des scénarios réalistes utilisateur : formulaires multi-étapes, popups, frames, et assertions conditionnelles. Les étudiants devront écrire leurs propres tests Playwright à partir des consignes.

\subsection*{Exercices}
\begin{enumerate}
\item \textbf{Formulaire multi-étapes}  
Vous devez automatiser la soumission d’un formulaire sur un site de démonstration. Le formulaire contient plusieurs champs obligatoires et des boutons \textit{Next} entre les étapes.  
\textbf{À faire :}  
\begin{itemize}
\item Remplir tous les champs obligatoires
\item Passer d’une étape à l’autre
\item Soumettre le formulaire
\item Vérifier l’apparition d’un message de confirmation
\end{itemize}

\item \textbf{Popup JavaScript}  
Certains sites affichent des alertes ou confirmations via JavaScript.  
\textbf{À faire :}  
\begin{itemize}
\item Cliquer sur un bouton déclenchant une alerte
\item Accepter ou rejeter l’alerte selon le scénario
\item Vérifier le texte affiché sur la page après l’alerte
\end{itemize}

\item \textbf{Interaction avec une iframe}  
Certains contenus sont intégrés dans des \textit{frames} ou \textit{iframes}.  
\textbf{À faire :}  
\begin{itemize}
\item Identifier la frame
\item Interagir avec un élément à l’intérieur (ex: remplir un champ ou cliquer sur un bouton)
\item Vérifier que l’action a bien été effectuée
\end{itemize}

\item \textbf{Assertions dynamiques et éléments conditionnels}  
Certains éléments peuvent apparaître ou disparaître dynamiquement après une action.  
\textbf{À faire :}  
\begin{itemize}
\item Cliquer sur un bouton qui modifie la présence d’un élément
\item Vérifier que l’élément a disparu ou réapparu
\item Vérifier le message affiché après l’action
\end{itemize}
\end{enumerate}

\subsection*{Questions de réflexion}
\begin{itemize}
\item Comment gérer les éléments qui apparaissent ou disparaissent dynamiquement ?
\item Quand utiliser une assertion stricte (\texttt{toHaveText}) ou partielle (\texttt{toContainText}) ?
\item Quels pièges pouvez-vous rencontrer avec les frames et comment les éviter ?
\item Comment rendre vos tests stables malgré des animations ou temps de chargement variables ?
\end{itemize}

---

\newpage
\section*{Partie C : Tests avancés, Page Object Model et Mock API}
\subsection*{But}
Structurer le projet avec le \textbf{Page Object Model}, créer des tests paramétrés et simuler des réponses réseau via mocking pour tester des scénarios complexes.

\subsection*{Exercices}
\begin{enumerate}
\item \textbf{Page Object Model}  
Vous devez créer une classe représentant la page de connexion d’un site de démonstration.  
\textbf{À faire :}  
\begin{itemize}
\item Identifier les champs de saisie et boutons importants
\item Créer des méthodes pour naviguer vers la page et se connecter
\item Ajouter des méthodes pour récupérer des messages d’erreur ou de succès
\end{itemize}

\item \textbf{Tests paramétrés}  
Vous devez écrire un test qui vérifie plusieurs combinaisons de nom d’utilisateur et mot de passe (correct et incorrect).  
\textbf{À faire :}  
\begin{itemize}
\item Paramétrer vos tests pour chaque scénario
\item Vérifier les messages d’erreur ou de succès correspondants
\item Vérifier l’URL après la connexion
\end{itemize}

\item \textbf{Mocking / Interception réseau}  
Certains tests nécessitent de simuler des réponses d’API pour tester des comportements précis.  
\textbf{À faire :}  
\begin{itemize}
\item Intercepter une requête réseau vers une API
\item Fournir une réponse simulée
\item Vérifier que les données mockées sont affichées correctement sur la page
\item Ajouter un test qui vérifie les modifications d’éléments après la réponse
\end{itemize}
\end{enumerate}

\subsection*{Questions de réflexion}
\begin{itemize}
\item Quels sont les avantages du Page Object Model dans un projet de tests complexe ?
\item Comment les tests paramétrés améliorent-ils la couverture fonctionnelle ?
\item Pourquoi et quand utiliser le mocking d’API dans un projet réel ?
\item Comment vérifier que le comportement de l’application est correct même avec des données simulées ?
\end{itemize}


\newpage
\section*{Partie D : Debugging, Reporting et CI/CD (bonus)}
\subsection*{But}
Découvrir les outils de debugging et configurer les rapports de test.

\subsection*{Exercices}
\begin{enumerate}
\item Capture d’écran et trace :
\begin{lstlisting}
// tests/debug.spec.ts
import { test } from '@playwright/test';

test('capture et trace', async ({ browser }) => {
  
});
\end{lstlisting}

\item Génération d’un rapport HTML

\item Configuration GitHub Actions :

\subsection*{Questions de réflexion}
\begin{itemize}
\item Comment analyser un fichier de trace avec Playwright ?
\item Quels avantages la CI apporte-t-elle à la qualité logicielle ?
\end{itemize}

\section*{Livrables}
\begin{itemize}
\item Code source complet du projet
\item Rapport HTML des tests exécutés
\item Fichier PDF contenant vos réponses aux questions
\item README expliquant comment exécuter les tests
\end{itemize}

\section*{Date de rendu}
Le TP est à rendre avant le \textbf{6 novembre 2025 à 20h00}.

\section*{Ressources utiles}
\begin{itemize}
\item Documentation officielle : \url{https://playwright.dev/docs/intro}
\item Exemples : \url{https://github.com/microsoft/playwright/tree/main/examples}
\item API Reference : \url{https://playwright.dev/docs/api/class-playwright}
\end{itemize}

\end{document}
