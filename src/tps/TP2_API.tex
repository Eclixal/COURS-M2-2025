\documentclass[a4paper,12pt]{article}
\usepackage[utf8]{inputenc}
\usepackage[T1]{fontenc}
\usepackage[margin=2.5cm]{geometry}
\usepackage[french]{babel}
\usepackage{hyperref}
\usepackage{listings}
\usepackage{xcolor}
\usepackage{fancyhdr}

\lstset{
    basicstyle=\ttfamily\small,
    keywordstyle=\color{blue},
    stringstyle=\color{red!70!black},
    commentstyle=\color{green!50!black},
    breaklines=true,
    showstringspaces=false
}

\pagestyle{fancy}
\fancyhf{}
\fancyfoot[C]{M2 MIASHS}
\fancyfoot[R]{\thepage}

\title{Travaux Pratiques \\ API Géographique avec Flask}
\author{Alexandre JOUSSET}
\date{\today}

\begin{document}

\maketitle

\section*{Introduction}

Flask est un micro-framework Python permettant de créer rapidement des APIs et applications web.  
Dans ce TP, vous allez construire pas à pas une \textbf{API géographique} capable de :
\begin{itemize}
    \item Retourner un message de bienvenue.
    \item Obtenir les coordonnées GPS d’une ville via OpenStreetMap.
    \item Calculer la distance entre deux villes avec OpenRouteService.
    \item (Bonus) Conserver l’historique des recherches.
\end{itemize}

\section*{Objectif}

\textbf{À la fin du TP}, vous aurez développé une API REST complète en Flask, intégrant des données externes et appliquant les bonnes pratiques de structuration et de documentation.

\section*{Matériel requis}

\begin{itemize}
    \item Python 3 installé.
    \item Bibliothèques : \texttt{flask}, \texttt{requests}.
    \item Un éditeur de code (VSCode, PyCharm, etc.).
\end{itemize}

\section*{Consignes générales}

\begin{itemize}
    \item Travaillez en binôme ou seul.
    \item Testez vos endpoints avec \texttt{curl}, Postman ou un navigateur.
    \item Gardez un code clair et commenté.
\end{itemize}

\section*{Exercices}

\subsection*{Exercice 1 : Installation et Hello World}

\begin{enumerate}
    \item Créez un dossier \texttt{api-geo}.
    \item Installez Flask.
    \item Créez un fichier \texttt{app.py} avec un endpoint \texttt{/} qui retourne un message de bienvenue.
    \item Lancez le serveur et testez l’URL dans un navigateur.
\end{enumerate}

\textbf{Question :} Pourquoi le mode \texttt{debug=True} est-il utile pendant le développement ?

\subsection*{Exercice 2 : Coordonnées d’une ville}

\begin{enumerate}
    \item Installez la bibliothèque \texttt{requests}.
    \item Ajoutez un endpoint \texttt{/coords/<city>} qui utilise l’API Nominatim d’OpenStreetMap pour retourner la latitude et la longitude de la ville passée en paramètre.
    \item Testez avec plusieurs villes (\texttt{/coords/Paris}, \texttt{/coords/Londres}, etc.).
\end{enumerate}

\textbf{Question :} Pourquoi faut-il préciser \texttt{format=json} dans la requête à Nominatim ?

\subsection*{Exercice 3 : Calculer une distance entre deux villes}

\begin{enumerate}
    \item Créez un compte sur \url{https://openrouteservice.org/} et récupérez une clé API gratuite.
    \item Créez un endpoint \texttt{/distance} qui prend en paramètres GET \texttt{from} et \texttt{to}.
    \item Utilisez Nominatim pour récupérer les coordonnées des deux villes.
    \item Utilisez OpenRouteService pour calculer la distance routière entre elles.
    \item Retournez la distance en kilomètres dans la réponse JSON.
\end{enumerate}

\textbf{Question :} Pourquoi est-il important de limiter le nombre d’appels ?

\subsection*{Exercice 4 : Améliorer la structure}

\begin{itemize}
    \item Déplacez la fonction qui récupère les coordonnées dans un fichier \texttt{utils.py}.
    \item Importez-la dans \texttt{app.py} pour éviter de répéter le code.
\end{itemize}

\textbf{Question :} Pourquoi est-il important de factoriser le code ?

\subsection*{Exercice 5 (Bonus) : Historique des recherches}

\begin{enumerate}
    \item Ajoutez une structure (liste ou autre) pour stocker chaque appel à \texttt{/distance} (ville de départ, ville d’arrivée, distance).
    \item Créez un endpoint \texttt{/history} qui retourne cet historique au format JSON.
    \item (Optionnel) Permettez de vider l’historique via un endpoint \texttt{DELETE}.
\end{enumerate}

\textbf{Question :} Comment rendre cet historique persistant même après redémarrage du serveur ?

\section*{Questions de réflexion}

\begin{enumerate}
    \item Quels sont les avantages de Flask par rapport à un framework plus lourd comme Django pour ce type de projet ?
    \item Quelles bonnes pratiques appliquer pour nommer et documenter ses endpoints ?
    \item Pourquoi limiter le nombre d’appels à des APIs externes ?
    \item Comment sécuriser une clé API dans un projet collaboratif ?
\end{enumerate}

\section*{Livrables}
\begin{itemize}
    \item Le code source complet de l’API.
    \item Un fichier \texttt{README.md} expliquant comment installer et utiliser l’API.
    \item Les réponses aux questions posées dans chaque exercice.
    \item La documentation des endpoints.
\end{itemize}

\section*{Date de rendu}
Le TP doit être rendu avant le \textbf{17 septembre 2025} à 18h.

\end{document}
